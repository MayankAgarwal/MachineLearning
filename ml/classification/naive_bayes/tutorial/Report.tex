% --------------------------------------------------------------
% This is all preamble stuff that you don't have to worry about.
% Head down to where it says "Start here"
% --------------------------------------------------------------
 
\documentclass[11pt]{article}
 
\usepackage[margin=1in]{geometry} 
\usepackage{amsmath,amsthm,amssymb}
\usepackage{setspace}
\usepackage{graphicx}
\usepackage{listings}
\usepackage{hyperref}
 
\newcommand{\N}{\mathbb{N}}
\newcommand{\Z}{\mathbb{Z}}

\usepackage{mathtools}
\DeclarePairedDelimiter{\ceil}{\lceil}{\rceil}
\DeclarePairedDelimiter{\floor}{\lfloor}{\rfloor}
 
\newenvironment{theorem}[2][Theorem]{\begin{trivlist}
\item[\hskip \labelsep {\bfseries #1}\hskip \labelsep {\bfseries #2.}]}{\end{trivlist}}
\newenvironment{lemma}[2][Lemma]{\begin{trivlist}
\item[\hskip \labelsep {\bfseries #1}\hskip \labelsep {\bfseries #2.}]}{\end{trivlist}}
\newenvironment{exercise}[2][Exercise]{\begin{trivlist}
\item[\hskip \labelsep {\bfseries #1}\hskip \labelsep {\bfseries #2.}]}{\end{trivlist}}
\newenvironment{problem}[2][Problem]{\begin{trivlist}
\item[\hskip \labelsep {\bfseries #1}\hskip \labelsep {\bfseries #2.}]}{\end{trivlist}}
\newenvironment{question}[2][Question]{\begin{trivlist}
\item[\hskip \labelsep {\bfseries #1}\hskip \labelsep {\bfseries #2.}]}{\end{trivlist}}
\newenvironment{corollary}[2][Corollary]{\begin{trivlist}
\item[\hskip \labelsep {\bfseries #1}\hskip \labelsep {\bfseries #2.}]}{\end{trivlist}}

\newenvironment{solution}{\begin{proof}[Solution]}{\end{proof}}
 
\begin{document}
 
% --------------------------------------------------------------
%                         Start here
% --------------------------------------------------------------
 
\title{Naive Bayes classifier}
\author{Mayank Agarwal}
\maketitle

\underline{Introduction}
\begin{enumerate}

\item Naive Bayes classifiers are a family of simple probabilistic classifiers that are based on applying Bayes theorem with strong (naive) assumptions of independence between the features.

\item Despite its strong assumptions of independence, this method remains quite popular in text categorization applications, and with proper pre-processing can compete against more sophisticated algorithms like SVM.

\item This is a highly scalable classifier since it requires a number of parameters linear in the number of features.

\item An advantage of Naive Bayes classifiers is that they require a small number of training examples to estimate the parameters necessary for classification.

\end{enumerate}

\underline{Probabilistic model}
\begin{enumerate}

\item Abstractly, naive Bayes is a conditional probability model, i.e., given an observation $x = (x_1, x_2, ..., x_n)$, it assigns to this instance probabilities:\\
$$p(C_k | x_1, ..., x_n)$$ \\
for each of K possible outcomes.\\
\\
Using Bayes theorem, this conditional probability can be written as: \\
$$p(C_k | x) = \frac{p(C_k) p(x | C_k)}{p(x)}$$\\
in Bayesian probability terminology, this is equivalent to $posterior = \frac{prior \times likelihood}{estimate}$\\

\item Note than the denominator in the above equation does not depend on $C_k$ and is constant for a given observation $x$. Therefore, we can ignore the \textit{estimate} in our calculations. 

\item The numerator is equivalent to the joint probability distribution $p(C_k, x_1, x_2, ..., x_n)$, which can be re-written as follows using the Chain rule. \\
$p(C_k, x_1, x_2, ..., x_n) = P(x_1, x_2, ..., x_n, C_k)

\end{enumerate}

\underline{References:}
\begin{enumerate}
\item Naive Bayes classifier. (2017, January 20). In Wikipedia, The Free Encyclopedia. Retrieved 01:02, February 22, 2017, from \url{https://en.wikipedia.org/w/index.php?title=Naive_Bayes_classifier&oldid=761016630}
\end{enumerate}

\end{document}
